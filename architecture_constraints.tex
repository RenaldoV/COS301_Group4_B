\section{Architecture Constraints}
\subsection{EE}
The client has requested that we use the Java EE environment if possible when developing a large portion of the Buzz System. \\

Systems developed using Java EE is known to be fairly resource intensive, which may result in performance issues. The effect of this constraint may be amplified due to the time (un-optimised code) and system resources constraints (low hardware performance). A large portion of the students who are going to be developing the system to do have experience with using Java EE and learning to use this environment will take up implementation time from the already tight development schedule.

\subsection{Time (Project deadlines)}
Deadlines are set for the delivery of the project for which different phases of the project should be demonstrated and evaluated by the client. \\

Time is a major constraint, because the deadlines that were given are non-negotiable. The development of the system needs to be completed in the allocated time or else the developers would have failed the assignment given to them by the client. This will inhibit the quality of the implementation as the programmers will have to follow a strict schedule and will focus largely on the throughput of their work, instead of ensuring that they write good, maintainable and reliable code. If the development team run into an unexpected delay one, or many, of the quality requirements will have to be compromised to remain within the time constraint bound.

\subsection{System Resources}
The University will most likely not spend a large amount of money on a powerful server for the system.\\

The hardware on which the system will be running places a bound on the performance the developers can expect from the final product. This has to be taken into consideration when deciding which quality requirements are realistic.
