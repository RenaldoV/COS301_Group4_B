\section{Quality Requirements}
\subsection{Critical}
\subsubsection{Scalability}
The system is expected to have a maximum of 2000 users as specified by the client. Buzz Space is meant to encourage discussions on multiple topics at the same time by a large amount of users so the system needs to be able to handle many concurrent users too.\\

\subsubsection{Monitor-ability and Audit-ability}

The system should be monitor-able and audit-able, to enable lecturers to monitor the discussion boards and to see who posted what on the Buzz Space. Lecturers should be able to evaluate and monitor each user to be able to give the user a participation mark for the discussion board.  Lecturers should be able to prevent irrelevant posts and discussions and should be able to see who were part of the discussion and notify them that it has been removed and should then be able to give them a warning.\\

This can all be achieved by ensuring that the system is monitor-able and audit-able.\\

\subsection{Important}

\subsubsection{Reliability and Availability}
Reliability and Availability is important in our system, as maintainability will then be less needed in an ideally reliable system. One of the key features of a discussion board is that it should be available 24/7. Users want answers and help as soon as possible, and therefore the system should be reliable and available at all times. If the system is very reliable and available, discussions that are of a time essence can be seen to as soon as possible. Reliability can be defined as the probability that a system will produce correct outputs up to some given time. A reliable system does not silently continue and deliver results that include uncorrected corrupted data. Availability means the probability that a system is operational at a given time. A highly available system would disable the malfunctioning portion and continue operating at a reduced capacity. \\

If the system is very reliable and available, users will be able to use the system with ease and the users will trust the Buzz system.\\

\subsubsection{Security}
Security is important because the evaluation of users' participation is involved. Not all of the users should be able to access these sensitive information and therefore certain users will be authorized to do certain actions, where other users will not be allowed to. Only users who are registered to the Buzz Space gave access and will be able to partake in the discussion boards. If the system is very secure, it will help to ensure Reliability and Availability.\\

\subsubsection{Usability}
The system must be as usable as possible so that users are encouraged to use it as much as possible. The extent to which the system can be made usable is affected by the priority of scalability and audit-ability because if the system is made scalable so that it can be used by all students at the same time, it is impossible to cater for 2000 student's needs and make it 100\% usable for the mass. Usability is marked important because the participation of students can be greatly affected if the system is unusable and students will then lose marks.\\

\subsubsection{Integrability}
One of the main functional requirements of the buzz system is plug-ability, this goes hand in hand with the quality requirement integrability thus marked as important. The system need to be able to plug into any other systems and also need to use other systems where it can.\\

\subsection{Nice to have}

\subsubsection{Testability}
Testability cannot be fully implemented or given a high priority because quality requirements, scalability and audit-ability have been given high priority and that affects the extent to which the system can be tested. Users will test the system and give their feedback.\\

\subsubsection{Performance}
Performance is expected to be reduced as scalability takes priority. When the system is being used by a large amount of people at one time it will be acceptable for it to take a few seconds to do anything. For small amounts of concurrent users the system should still be fast.\\

\subsubsection{Maintainability}
Once the system is implemented it is not expected to need regular maintenance as reliability is an important quality requirement.\\







