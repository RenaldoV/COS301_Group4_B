%If working on sections in "1.1 Architecture requirements" only use from subsection downwards

\section{Architectural patterns}
	\subsection{Loose Coupling}
		\subsubsection{Description}
		Loose coupled systems only allow accessing of resources through a proxy or accessing a pool of resources instead of a single resource. Loose coupling refers to the indirect access to external resources. 
		\subsubsection{Quality Requirements it enforces}
		\begin{itemize}
			\item Scalability \\
			This loose coupling of systems via API's makes your application more scalable because systems can off-load part of your processing to other systems that are independently scalable. 
			\item Maintainability \\
			This loose coupling of systems via API's makes your application more maintainable because bugs and problems are enclosed in the module and not in the system as a whole. 
			\item Integration \\
			Because loose coupled modules should work independent, it's very easy to integrate a loose coupled system into another system by using an adapter module. 
			\item Security \\
			Because loose coupled modules are only accessed through the API, all databases/servers are only accessed by the module and not directly by the user.
		\end{itemize}	
			
	\subsection{Blackboard Pattern}
		\subsubsection{Description}
 By allowing multiple processes such as creating a root thread or rating a user to work closer together on different threads, polling and reacting if needed so that less time is wasted by running processes in parallel.
		\subsubsection{Quality Requirements it enforces}
		\begin{itemize}
			\item Scalability\\
It is easy to achieve scalability with the blackboard pattern accross a grid of processors subject to having a scalable implementation of the blackboard itself.
		\end{itemize}	
		
	\subsection{Prototype pattern}
		\subsubsection{Description}
 			By deep cloning objects to store their data in the audit log if any changes occurred, without keeping a reference to the original data and exposing it.
		\subsubsection{Quality Requirements it enforces}
		\begin{itemize}
			\item Maintainability and Auditability\\
The prototype pattern makes clones of the old values(deep cloning) and then extracts the data needed in the audit trail.
			\item Security\\
Because the prototype pattern works on a skeleton based design in some sense it helps with security by limiting manipulation possibilities.

		\end{itemize}	